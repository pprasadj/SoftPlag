\chapter{Analysis and Design}
\section{Methodology / Procedure adopted}
\begin{itemize}
\item The common ways adopted by people for plagiarizing the code are:
\begin{enumerate}
 \item The original source code can be replicated as it is.
 \item Addition of comments in the source code.
 \item Modification in identifiers. 
 \item Chane in variable position.
 \item The procedure combination can be done in source code.
 \item Program statements can be changed by some modifications.
 \item Control Logic can be modified.
 \end{enumerate}
\item Describe on the development methodology / model you would use. (E.g. Agile method or Iterative Model)
\begin{enumerate}
\item In Iterative model, iterative process starts with a simple implementation of a small set of the software requirements and iteratively enhances the evolving versions until the complete system is implemented and ready to be deployed.
\item An iterative life cycle model does not attempt to start with a full specification of requirements. Instead, development begins by specifying and implementing just part of the software, which is then reviewed in order to identify further requirements. This process is then repeated, producing a new version of the software at the end of each iteration of the model.\\
\end{enumerate}
\item How you intend manage the weekly meetings ? \\
The weekly meetings need to managed properly because in order to accomplish the goals desired, you will need to have a good strategic and tactical plan. In the meeting, plans may be decided by each team member and the procedure is been planned.
\item How do you intend to monitor and measure the progress of the project? \\ 
The monitoring and measure of progress of report is been done on basis of different modules. A schedule is maintained for each module to be complemented. Github is used for project monitoring ,as all the team members upload their work on completion. 
\end{itemize}
\section{Analysis}

Based on the requirements gathered, how was the feasibility study of the project carried out? \\
The project is about the detection of software source code plagiarism ,so the study carried out on the comparison of the different plagiarism software's which are based on rule based algorithms.\\
On comparison of different features of some software's.\\
The papers were referred for plagiarism detection, which gives the idea of different levels of plagiarism and defining the metrics for the programs.  \\
If any requirements, were modified why they were modified? \\

\subsection{Software / System Requirement Specification - IEEE format}
\textbf{Table of Contents}\\
\textbf{Revision History}\\
\textbf{1.\hspace{0.3cm}	Introduction}\\
\textbf{1.1\hspace{0.3cm} Purpose}\\
\textbf{1.2 \hspace{0.3cm} Document Conventions}\\
\textbf{1.3 \hspace{0.3cm} Intended Audience and Reading Suggestions}\\
\textbf{1.4 \hspace{0.3cm} Product Scope}\\
\textbf{1.5 \hspace{0.3cm} References}\\
\textbf{2.\hspace{0.3cm}	Overall Description}\\
\textbf{2.1	\hspace{0.3cm}Product Perspective}\\
\textbf{2.2\hspace{0.3cm}	Product Functions}\\
\textbf{2.3	\hspace{0.3cm}User Classes and Characteristics}\\
\textbf{2.4	\hspace{0.3cm}Operating Environment}\\
\textbf{2.5\hspace{0.3cm}	Design and Implementation Constraints}\\
\textbf{3.	\hspace{0.3cm}External Interface Requirements}\\
\textbf{3.1	\hspace{0.3cm}User Interfaces}\\
\textbf{3.2	\hspace{0.3cm}Hardware Interfaces}\\
\textbf{3.3	\hspace{0.3cm}Software Interfaces}\\
\textbf{4.\hspace{0.3cm}	System Features}\\
\textbf{4.1	\hspace{0.3cm}System Feature 1}\\
\textbf{5.\hspace{0.3cm}	Other Nonfunctional Requirements}\\
\textbf{5.1	\hspace{0.3cm}Software Quality Attributes}\\
\textbf{5.2	\hspace{0.3cm}Business Rules}\\
\textbf{Appendix A: Glossary}\\
\textbf{Appendix B: Analysis Models}\\
\textbf{Appendix C: To Be Determined List}\\\\

\textbf{\huge Revision History}\\

\begin{center}
\begin{tabular}{ |c|c|c|c| } 

 \hline
 Name & Date & Reason For Changes & Version \\ [0.5ex] 
  \hline
  &  &  &  \\ 
  \hline
  &  &   & \\ 
  \hline
\end{tabular}
\end{center}
\textbf{1.	Introduction}\\
 Plagiarism Detection is the process of finding  instances of plagiarism within a work or document. The worldwide and widespread use of computers and  Internet has made it easier to plagiarize the work of others. Most cases of plagiarism are found in academia, where source code, art designs, documents like essays , reports etc. So Our Software is use to find plagiarism in source code.\\
In our plagiarism detection software detection of plagiarism is software-assisted. Software-assisted detection allows vast collections of documents to be compared to each other, making successful detection much more likely. Software assisted reduces effort and time required for plagiarism detection. \\
\textbf{1.1	Purpose }\\
Plagiarism means  the practice of taking someone else's work or ideas and passing them off as one's own. Plagiarism detection tool is useful to detect plagiarism. Plagiarism are found in colleges, organization or any institute level. So our software is useful for colleges ,Organizations, Institutes to find plagiarism in their respective work area. Our software tool  only use to find plagiarism in source code. Our Aim to reduce plagiarism in all field. so everyone will look to think and present their idea or work instead of showing others ideas or work.\\
\textbf{1.2	Document Conventions}\\
// <Describe any standards or typographical conventions that were followed when writing this SRS, such as fonts or highlighting that have special significance. For example, state whether priorities  for higher-level requirements are assumed to be inherited by detailed requirements, or whether every requirement statement is to have its own priority.>\\
\textbf{1.3	Intended Audience and Reading Suggestions}\\
This Document is intended for developers ,designers ,tester ,project managers, users and documentation writers. Document contains scope of product ,product Perspective ,functions, hardware and software specification ,features of product.\\
\textbf{1.4	Product Scope}\\
Plagiarism detection software take user input as an java file and it checks  with  repository present on local machine and display result. Scope  of product is only java file can be check to decide it  is plagiarized or not. The source  and target program should be in Java.\\
\textbf{1.5	References}\\
// 
<List any other documents or Web addresses to which this SRS refers. These may include user interface style guides, contracts, standards, system requirements specifications, use case documents, or a vision and scope document. Provide enough information so that the reader could access a copy of each reference, including title, author, version number, date, and source or location.>\\\\
\textbf{2.	Overall Description}\\
\textbf{2.1	Product Perspective}\\
Plagiarism detection software is used to check plagiarism in source code in java language only. It is new ,self-contained product. It will able to detect all levels of plagiarism in java programming language. It will implement and extends all features that are not present in other tool and required for plagiarism detection. What Metrics we are  consider to check whether file is plagiarized or not those all metrics are important parameter of our product.  It will help to reduce plagiarism in an institutes, colleges and organizations.\\
\textbf{2.2	Product Functions}\\
In Plagiarism detection software from user system will take java program file to check whether it is plagiarized or not. System will compare it with program in repository and if it is match then it will show that file it plagiarized and if it does not match then system shows file is not plagiarized. Main function of it that it will detect all levels of plagiarism. Effective for detecting plagiarized file. \\
\textbf{2.3	User Classes and Characteristics}\\
//
<Identify the various user classes that you anticipate will use this product. User classes may be differentiated based on frequency of use, subset of product functions used, technical expertise, security or privilege levels, educational level, or experience. Describe the pertinent characteristics of each user class. Certain requirements may pertain only to certain user classes. Distinguish the most important user classes for this product from those who are less important to satisfy.>\\
\textbf{2.4	Operating Environment}\\
Software Requirement :\\
Java Runtime Environment\\
Operating environment –Windows /Linux\\\\
Hardware Requirement:\\
Minimum memory – 512 Mb  RAM\\
Hard disk space-1 GB\\
\textbf{2.5	Design and Implementation Constraints}\\
Project  Scope is only for java language so to work this tool properly both user file and files which are present in repositories should be  in java .\\\\
\textbf{3.	External Interface Requirements}\\
\textbf{3.1	User Interfaces}\\
There will be an screen where user has to upload file for plagiarism detection. So will give an upload button to user to upload any java file an then will compare it with our repository  and then display result about it. Result will be shown in an percentage format based on levels of plagiarism.\\
\textbf{3.2	Hardware Interfaces}\\
For this software minimum 512 Mb RAM and 1 GB hard disk is required. It can be use in windows or linux operating systems.\\
\textbf{3.3	Software Interfaces}\\
Plagiarism detection software contain repositories which has no of java files. So whenever user gives an input it will compare it with repository files and display result.\\\\
\textbf{4.	System Features}\\
\textbf{4.1	Level wise Plagiarism Detection }\\
\textbf{4.1.1	Description and Priority}\\
In our system it will detect plagiarism in user file level by level by  checking all  plagiarism levels conditions.\\
Level wise plagiarism detection feature has high priority in our project. \\
\textbf{4.1.2  Stimulus/Response Sequences}\\
Here input from user will take like any java file then system will process it. compare With it repository including all levels of plagiarism. Then result will be display to the user.\\
\textbf{4.1.3	Functional Requirements}\\
To work all functions of the system properly on user system. Java runtime environment should be installed on user machine to run software and to use all functionality . \\\\
\textbf{5.	Other Nonfunctional Requirements}\\
\textbf{5.1	Software Quality Attributes}\\
\textbf{Correctness:} When user give an input of an java file to check whether the file is plagiarized or not then system will shows an correct result to user by comparing it with an repository files.\\
\textbf{Reliability:} This will be an simple application even work in an minimum ram and space so it will not crash .\\
\textbf{Performance :}The performance of the system will be optimal. It will be easy to operate for an users.\\
\textbf{Usability:} We will make an software as simple as possible for the users understanding .So it is easy to use an software to any user.\\
\textbf{5.2	Business Rules}\\
//
<List any operating principles about the product, such as which individuals or roles can perform which functions under specific circumstances. These are not functional requirements in themselves, but they may imply certain functional requirements to enforce the rules.>\\\\




\section{Proposed System}

Give the details of your proposed system and architecture 
Advantage of the proposed system over the existing system

\subsection{Hardware / Software requirements}
Development Hardware / Software requirements \\\\
Software Requirements : \\
Java  JDK 1.4 or more \\
Operating System : Window/Linux\\
Front-End : Java Fx\\
Back-End : MySQL\\\\
Hardware Requirements :\\
Minimum Memory : 512 Mb RAM \\
Hard disk space : 1 GB\\\\
Deployment Hardware / Software requirements \\\\
Software Requirements : \\
Java Runtime Environment\\
Operating System : Window/Linux. \\\\
Hardware Requirements : \\
Minimum Memory : 512 Mb RAM \\
Hard disk space : 1 GB\\

\subsection{Design Details}

Different UML diagrams as per the project requirement (For e.g. Use Case Diagram)

\subsection{Implementation Plan}

Timeline chart is for Next semester 





\section{Proposed System}

Give the details of your proposed system and architecture 
Advantage of the proposed system over the existing system

\subsection{Hardware / Software requirements}
Development Hardware / Software requirements \\
Deployment Hardware / Software requirements \\

\subsection{Design Details}

Different UML diagrams as per the project requirement (For e.g. Use Case Diagram)

\subsection{Implementation Plan}

Timeline chart is for Next semester
