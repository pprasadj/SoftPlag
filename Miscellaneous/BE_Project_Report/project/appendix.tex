	\begin{center}
	\thispagestyle{empty}
	\vspace{2cm}
	\LARGE{\textbf{Appendix - I}}\\[1.0cm]
	\end{center}
	
	\textbf{Software / System Requirement Specification - IEEE format}\\
	\textbf{Table of Contents}\\
	1.\hspace{0.3cm}	Introduction\\
	1.1\hspace{0.3cm} Purpose\\
	1.2 \hspace{0.3cm} Document Conventions\\
	1.3 \hspace{0.3cm} Intended Audience and Reading Suggestions\\
	1.4 \hspace{0.3cm} Product Scope\\
	1.5 \hspace{0.3cm} References\\
	2.\hspace{0.3cm}	Overall Description\\
	2.1	\hspace{0.3cm}Product Perspective\\
	2.2\hspace{0.3cm}	Product Functions\\
	2.3	\hspace{0.3cm}User Classes and Characteristics\\
	2.4	\hspace{0.3cm}Operating Environment\\
	2.5\hspace{0.3cm}	Design and Implementation Constraints\\
	3.	\hspace{0.3cm}External Interface Requirements\\
	3.1	\hspace{0.3cm}User Interfaces\\
	3.2	\hspace{0.3cm}Hardware Interfaces\\
	3.3	\hspace{0.3cm}Software Interfaces\\
	4.\hspace{0.3cm}	System Features\\
	4.1	\hspace{0.3cm}System Feature 1\\
	5.\hspace{0.3cm}	Other Nonfunctional Requirements\\
	5.1	\hspace{0.3cm}Software Quality Attributes\\
	5.2	\hspace{0.3cm}Business Rules\\
	
	\textbf{1.	Introduction}\\
	 Plagiarism Detection is the process of finding  instances of plagiarism within a work or document. The worldwide and widespread use of computers and  Internet has made it easier to plagiarize the work of others. Most cases of plagiarism are found in academia, where source code, art designs, documents like essays , reports etc. So Our Software is use to find plagiarism in source code.\\
	In our plagiarism detection software detection of plagiarism is software-assisted. Software-assisted detection allows vast collections of documents to be compared to each other, making successful detection much more likely. Software assisted reduces effort and time required for plagiarism detection. \\
	\textbf{1.1	Purpose }\\
	Plagiarism means  the practice of taking someone else's work or ideas and passing them off as one's own. Plagiarism detection tool is useful to detect plagiarism. Plagiarism are found in colleges, organization or any institute level. So our software is useful for colleges ,Organizations, Institutes to find plagiarism in their respective work area. Our software tool  only use to find plagiarism in source code. Our Aim to reduce plagiarism in all field. so everyone will look to think and present their idea or work instead of showing others ideas or work.\\
	
	\textbf{1.3	Intended Audience and Reading Suggestions}\\
	This Document is intended for developers ,designers ,tester ,project managers, users and documentation writers. Document contains scope of product ,product Perspective ,functions, hardware and software specification ,features of product.\\
	\textbf{1.4	Product Scope}\\
	Plagiarism detection software take user input as an java file and it checks  with  repository present on local machine and display result. Scope  of product is only java file can be check to decide it  is plagiarized or not. The source  and target program should be in Java.\\
	\textbf{1.5	References}\\
	 
	 
	 \begin{itemize}
	 
	\item Fonte, V. Martins, R. Henriques and D. da Cruz, 1st ed. Portugal, 1998, pp. 147-148.\\
	
	
	\item Cosma, Georgina. A Thesis Submitted In Partial Fulfilment Of The Requirements For The Degree Of Doctor Of Philosophy In Computer Science. 1st ed. Warwick: N.p., 2008. Print.\\
	
	\item Parker, Alan and James Hamblen. Computer Algorithms For Plagiarism Detection. 2nd ed. Atlanta: N.p., 1989. Print.\\
	\item Ali, Asim M. El Tahir and Vaclav Snase. Overview And Comparison Of Plagiarism Detection Tools. 1st ed. Czech Republic: N.p. Web.\\
	\item MUFTAH, AHMED JABR AHMED. Document Plagiarism Detection algorithm using semantic network. 1st ed. Malaysia: N.p., 2009. Print.\\
	\item Haider, Khurram Zeeshan, Tabassam Nawaz, and Ali Javed. Efficient Source Code Plagiarism Identification Based On Greedy String Tilling. 10th ed. Taxila: N.p., 2010. Web. 23 Aug. 2016.\\
	
	\end{itemize}
	\textbf{2.	Overall Description}\\
	\textbf{2.1	Product Perspective}\\
	Plagiarism detection software is used to check plagiarism in source code in java language only. It is new ,self-contained product. It will able to detect all levels of plagiarism in java programming language. It will implement and extends all features that are not present in other tool and required for plagiarism detection. What Metrics we are  consider to check whether file is plagiarized or not those all metrics are important parameter of our product.  It will help to reduce plagiarism in an institutes, colleges and organizations.\\
	\textbf{2.2	Product Functions}\\
	In Plagiarism detection software from user system will take java program file to check whether it is plagiarized or not. System will compare it with program in repository and if it is match then it will show that file it plagiarized and if it does not match then system shows file is not plagiarized. Main function of it that it will detect all levels of plagiarism. Effective for detecting plagiarized file. \\
	\textbf{2.4	Operating Environment}\\
	Software Requirement :\\
	Java Runtime Environment\\
	Operating environment –Windows /Linux\\
	Hardware Requirement:\\
	Minimum memory – 512 Mb  RAM\\
	Hard disk space-1 GB\\
	\textbf{2.5	Design and Implementation Constraints}\\
	Project  Scope is only for java language so to work this tool properly both user file and files which are present in repositories should be  in java .\\
	\textbf{3.	External Interface Requirements}\\
	\textbf{3.1	User Interfaces}\\
	There will be an screen where user has to upload file for plagiarism detection. So will give an upload button to user to upload any java file an then will compare it with our repository  and then display result about it. Result will be shown in an percentage format based on levels of plagiarism.\\
	\textbf{3.2	Hardware Interfaces}\\
	For this software minimum 512 Mb RAM and 1 GB hard disk is required. It can be use in windows or linux operating systems.\\
	\textbf{3.3	Software Interfaces}\\
	Plagiarism detection software contain repositories which has no of java files. So whenever user gives an input it will compare it with repository files and display result.\\
	\textbf{4.	System Features}\\
	\textbf{4.1	Level wise Plagiarism Detection }\\
	\textbf{4.1.1	Description and Priority}\\
	In our system it will detect plagiarism in user file level by level by  checking all  plagiarism levels conditions.\\
	Level wise plagiarism detection feature has high priority in our project. \\
	\textbf{4.1.2  Stimulus/Response Sequences}\\
	Here input from user will take like any java file then system will process it. compare With it repository including all levels of plagiarism. Then result will be display to the user.\\
	\textbf{4.1.3	Functional Requirements}\\
	To work all functions of the system properly on user system. Java runtime environment should be installed on user machine to run software and to use all functionality . \\
	\textbf{5.	Other Nonfunctional Requirements}\\
	\textbf{5.1	Software Quality Attributes}\\
	\textbf{Correctness:} When user give an input of an java file to check whether the file is plagiarized or not then system will shows an correct result to user by comparing it with an repository files.\\
	\textbf{Reliability:} This will be an simple application even work in an minimum ram and space so it will not crash .\\
	\textbf{Performance :}The performance of the system will be optimal. It will be easy to operate for an users.\\
	\textbf{Usability:} We will make an software as simple as possible for the users understanding .So it is easy to use an software to any user.\\
